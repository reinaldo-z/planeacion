\documentclass[letterpaper,10pt]{article}
\usepackage[hmargin=3.5cm, vmargin=2.25cm]{geometry} 
\usepackage{color,graphicx}
\usepackage{eso-pic}
\usepackage[usenames,dvipsnames]{xcolor}
% \usepackage{xunicode}
% \usepackage{fontspec}
\usepackage{parskip}
\usepackage[absolute]{textpos}
\usepackage[spanish]{babel}
\spanishdecimal{.}

\usepackage{fullpage}

\newcommand\BackgroundLogo{
\put(175,320){
\parbox[b][\paperheight]{\paperwidth}{%
\vfill
\centering
\includegraphics[width=4cm,height=2cm,keepaspectratio]{/Users/reinaldo/Documents/clases/jassa/logo}
\vfill
}}}

\usepackage{array}
\newcolumntype{L}[1]{>{\raggedright\let\newline\\\arraybackslash\hspace{0pt}}m{#1}}
\newcolumntype{C}[1]{>{\centering\let\newline\\\arraybackslash\hspace{0pt}}m{#1}}
\newcolumntype{R}[1]{>{\raggedleft\let\newline\\\arraybackslash\hspace{0pt}}m{#1}}

%-----------------------------Fonts---------------------------------%
% \defaultfontfeatures{Mapping=tex-text}
% \setmainfont[Mapping=tex-text]{Linux Libertine O}     % http://www.linuxlibertine.org/
% \setsansfont[Scale=MatchLowercase,Mapping=tex-text]{Linux Biolinum O} % http://www.linuxlibertine.org/
% \setmonofont[Scale=MatchLowercase]{FreeMono}          % http://www.gnu.org/software/freefont/

%--------------------Link colors and pdf info-----------------------%
\usepackage{hyperref}
\definecolor{headings}{HTML}{701112}
\definecolor{links}{rgb}{0,0.2,0.6}
\hypersetup{colorlinks,breaklinks,urlcolor=links,linkcolor=links,pdftitle=Planeaci\'on bimestral -- Reinaldo A. Zapata Pe\~na,pdfauthor=Reinaldo A. Zapata Pe\~na}

%-----------------------------Watermark-----------------------------%
\setlength{\TPHorizModule}{30mm}
\setlength{\TPVertModule}{\TPHorizModule}
\textblockorigin{4mm}{0.2\paperheight}
\setlength{\parindent}{0pt}

%----------------------Custom section headings----------------------%
\usepackage{titlesec}
\titleformat{\section}
{\color{headings}\scshape\sffamily\Large\raggedright}{}{0em}{}[\color{black}\titlerule]
\titlespacing{\section}{0pt}{3pt}{3pt}

\title{\sc \Huge Secuencia Did\'actica Bimestral \\ Primero de Secundaria}

\begin{document}
\AddToShipoutPicture*{\BackgroundLogo}
\ClearShipoutPicture

\date{}

\maketitle

\vspace{-2.3cm}

\begin{center}
{\Large\color{headings}\sffamily{Secundaria Incorporada a la SEP  }}
\end{center}

\vspace{5mm}

\section{Informaci\'on general}
\begin{tabular}{R{0.25\textwidth}p{0.71\textwidth}}
    \textsc{Docente:}       &   M. en C. Reinaldo Arturo Zapata Pe\~na. \\
    \textsc{Materia:}       &   Matem\'aticas, primero de secundaria. \\
    \textsc{Grupos:}        &   A, C. \\
    \textsc{Ciclo escolar:} &   2015 -- 2016. \\
    \textsc{Periodo:}       &   Primer bimestre, 4 de agosto -- 19 de octubre de 2015. \\
                            &   Un total de 45 sesiones de 50 minutos. \\
    \textsc{Criterios de}   &   Examen parcial:     35\%. \\   
    \textsc{evaluaci\'on:}  &   Examen bimestral:   35\%. \\
                            &   Tareas y trabajos:  25\%. \\
                            &   Participaci\'on:     5\%. \\
    \textsc{Seguimiento:}   &   Se mandar\'an avisos a casa para los padres de
    aquellos alumnos que no entreguen dos o m\'as tareas. \\

\end{tabular}


\section{Competencias a trabajar}
\begin{tabular}[t]{R{0.25\textwidth}p{0.71\textwidth}}
    \textsc{Competencias: } &   Resolver problemas de manera aut\'onoma. \\ 
                            &   Comunicar información matem\'atica. \\
                            &   Validar procedimientos y resultados. \\
                            &   Manejar t\'ecnicas eficientemente. \\
\end{tabular}



\section{Temas  y secuencias did\'acticas}
\begin{tabular}[t]{R{0.25\textwidth}p{0.71\textwidth}}
    \textsc{Temas y subtemas:}          & Presentaci\'on del profesor y de las reglas generales en el sal\'on de clase. \\
    \textsc{Tiempo:}                    & 1 sesi\'on, 4 de agosto. \\
\\ \hline \\
\end{tabular}


\begin{tabular}[t]{R{0.25\textwidth}p{0.71\textwidth}}
    \hline \\

    \textsc{Tiempo:}                    & 3 sesiones, 24 -- 27 de agosto. \\
    \textsc{Temas y subtemas:}          & N\'umeros primos y m\'inimo com\'un m\'utiplo. \\
    \textsc{Aprendizaje esperado: }     & El alumnos aprende la definici\'on de
    n\'umero primo y los identifica. Usando n\'umeros primos descompone
    n\'umeros compuestos en factores primos para obtener as\'i obtener el
    m\'inimo com\'un m\'ultiplo de un grupo de n\'umeros. \\
    \textsc{Evidencias a evaluar:}      & Un trabajo en clase y una tarea. \\\\
    \textsc{\large Secuencias:} \\
    Profesor    & Explica el concepto de número primo. Pide a los alumnos que
    busquen la definici\'on en le diccionario. \\ Alumno:     & Habiendo
    obtenido la definici\'on de n\'umero primo revisa los primeros
    correspondientes del n\'umero 2 al 30.\\
    Profesor:   & Explica el procedimiento para hacer la descomposici\'on en factores primos de un grupo de n\'umeros para asi obtener el m\'inimo com\'un m\'ultiplo (MCM) de los mimsmos. \\
    Alumno:     & Usa el procedimiento para obtener el MCM de una serie de n\'umeros. 
    
\\ \hline \\

    \textsc{Tiempo:}                    & 1 sesi\'on, 28 de agosto. \\
    \textsc{Temas y subtemas:}          & Aplicaci\'on de examen diagn\'ostico. \\
    
\\ \hline \\
    
    \textsc{Tiempo:}                    & 5 sesiones, 31 de agosto -- 4 de septiembre. \\
    \textsc{Temas y subtemas:}          & Fracciones equivalentes, criterios de
    divisibilidad y simplificaci\'on de fracciones. \\
    \textsc{Aprendizaje esperado: }     & El alumno aprende le concepto de
    fracci\'on equivalente y lo aplica a un grupo de fracciones dadas. Aprende
    los criterios de divisibilidad para los n\'umeros 2, 3, 4, 5, 6, 8, 9, y 10.
    Utilizando dichos criterios de divisibilidad aprende a simplificar
    fracciones propias hasta su m\'imima expresi\'on. \\
    \textsc{Evidencias a evaluar:}      & Dos trabajos en clase y una tarea. \\\\
    \textsc{\large Secuencias:} \\
    Profesor:       & Explicaci\'on del concepto de equivalencia y su
    aplicaci\'on a fracciones. Ejemplos de fracciones equivalentes.\\ Alumno: &
    Ejercicios de fracciones equivalents a partir de fracciones dadas. \\ Ambos: 
    & Revisi\'on grupal de resultados. \\ 

\\ \hline \\

    \textsc{Tiempo:}                    & 5 sesiones, 7 -- 11 de septiembre. \\
    \textsc{Temas y subtemas:}          & M\'inimo com\'un m\'ultiplo (MCM) y
    fracciones equivalentes con mismo denominador. \\
    \textsc{Aprendizaje esperado: }     & El alumnos aprende a obtener el MCM de
    un grupo dado de n\'umeros. Usando el concepto de fracci\'on equivalente y
    el MCM convierte un grupo de fracciones a fracciones equivalentes con mismo
    denominador. \\
    \textsc{Evidencias a evaluar:}      & Dos trabajos en clase, dos tareas y un examen r\'apido. \\\\
    \textsc{\large Secuencias:} \\
    Profesor: & Explicaci\'on de la descomposici\'on en factores primos de una
    serie de n\'umeros para obtener el MCM.
\end{tabular}

\newpage

\begin{tabular}[t]{R{0.25\textwidth}p{0.71\textwidth}}
    Alumno: & Calcula el MCM de distintos grupos de n\'umeros. \\ Profesor: &
    Recuerda a los alumnos el concepto de fracci\'on equivalente. Explicaci\'on
    del proceso para convertir un grupo de fracciones a fracciones equivalentes
    con mismo denominador. Varios ejemplos resueltos. \\ Alumno: & Convierte
    fracciones a fracciones equivalentes con mismo denominadr. \\ Profesor: &
    Despeque de dudas. \\ Ambos: & Revisi\'on grupal de ejercicios y
    correcci\'on de errores. \\ Alumno: & Resoluci\'on de un examen r\'apido (10
    incisos) el d\'ia 11 de septiembre. \\ Profesor: & Retroalimentaci\'on
    r\'apida acerca de los resultados del examen el d\'ia lunes 14 de
    septiembre. \\

\\ \hline \\
 
    \textsc{Tiempo:}& 1 d\'ia, mi\'ercoles 16 de septiembre. \\
                    & {\Large \sc Suspensi\'on de labores. } \\ 
\\ \hline \\
    
    \textsc{Tiempo:}                    & 4 sesiones, 14 -- 18 de septiembre
    (una sesi\'on es perdida por el dia de asueto). \\
    \textsc{Temas y subtemas:}          & Suma y resta de fracciones con
    diferente denomiandor. \\
    \textsc{Aprendizaje esperado: }     & Utilizando el concepto de fracciones
    equivalentes con mismo denominador el alumno realiza operaciones de suma y
    resta de fracciones con diferente denominador. \\
    \textsc{Evidencias a evaluar:}      & Cuatro trabajos y dos tareas. \\\\
    \textsc{\large Secuencias:} \\
    Profesor: & Recordatorio del concepto de fracci\'on equivalente con mismo
    denominador. Explicaci\'on del proceso para sumar o restar fracciones con
    diferente denominador convirti\'endolas a fracciones equivalentes con mismo
    denominador. \\ Alumno: & Resoluci\'on de ej ercicios. \\ Profesor: &
    Despeje de dudas. \\ Ambos: & Revisi\'on grupal de ejercicios. \\ Profesor: &
    Retroalimentaci\'on de resultados en tareas y preparaci\'on para examen parcial. \\
    

\\ \hline \\
 
    \textsc{Tiempo:}& 2 d\'ias, lunes 21 y martes 22  de septiembre. \\
                    & {\Large \sc Aplicaci\'on de examen parcial} (21 de septiembre). \\ 
                    & Revisi\'on grupal del los resultados del examen parciall (22 de septiembre).\\
\\ \hline \\
 
    \textsc{Tiempo:}& 1 d\'ia, viernes 25 de septiembre. \\
                    & {\Large \sc Consejo T\'ecnico Escolar.} \\ 
\\ \hline \\

    \textsc{Tiempo:}                    & 3 sesiones, 23 -- 24, 28 de septiembre. \\
    \textsc{Temas y subtemas:}          & Multiplicaci\'on y divisi\'on con fracciones. \\
    \textsc{Aprendizaje esperado: }     & El alumno aprende el procedimiento
    para hacer operaciones de multiplicaci\'on y divisi\'on con n\'umeros
    fraccionarios. \\
    \textsc{Evidencias a evaluar:}      & Dos trabajos y dos tareas \\\\

\end{tabular}

\begin{tabular}[t]{R{0.25\textwidth}p{0.71\textwidth}}

    \textsc{\large Secuencias:} \\
    Profesor: & Explicaci\'on del proceso para realizar operaciones de
    multiplicaci\'on con fraciones. Resoluci\'on de ejemplos y despeje de dudas.
    \\ Alumno: & Resoluci\'on de ejercicios de multiplicaci\'on con fracciones.
    \\ Ambos: & Revisi\'on grupal de ejercicios. \\ Alumno: & correcci\'on de
    errores en caso de haberlos. \\ Profesor: & retroalimentaci\'on de
    ejercicios de tarea. \\ Profesor: & Explicaci\'on del proceso para realizar
    operaciones de divisi\'on con fraciones. Resoluci\'on de ejemplos y despeje
    de dudas. \\ Alumno: & Resoluci\'on de ejercicios de divisi\'on con
    fracciones. \\ Ambos: & Revisi\'on grupal de ejercicios. \\ Alumno: &
    Correcci\'on de errores en caso de haberlos. \\ Profesor: &
    Retroalimentaci\'on de ejercicios de tarea.

\\ \hline \\

    \textsc{Tiempo:}                   & 5 sesiones, 29 de septiembre -- 2 de octubre y 5 --6 de octubre. \\
    \textsc{Temas y subtemas:}         & Jerarq\'ia de las operaciones y signos de agrupaci\'on. \\
    \textsc{Aprendizaje esperado: }    & El alumno aprende la jerarq\'ia de las
    operaciones y el orden en que se tienen que realizar. Aprende adem\'as a
    utilizar signos de agrupaci\'on, tales como par\'entesis, corchetes y
    llaves, para realizar operaciones con fracciones; comprende que el mismo
    procedimiento se aplica a n\'umeros enteros y decimales. \\
    \textsc{Evidencias a evaluar:}      & Dos trabajos, dos tareas y un examen r\'apido. \\\\
    \textsc{\large Secuencias:} \\
    Profesor: & Puesta en conflicto a los alumnos al pedirles que realicen una
    operaci\'on en la que se mezclan sumas y restas con multiplicaciones y
    divisiones. Explicaci\'on de la jerarqu\'ia de las operaciones y el orden en
    que se deben hacer. \\ Alumno: & Comprensi\'on de la jerarqu\'ia de las
    operaciones y del orden en que se deben realizar. \\ Profesor: & Despeje de
    dudas. \\ Alumno: & Resoluci\'on de ejercicios y tareas. \\ Ambos: &
    Revisi\'on grupal de trabajos. \\ Profesor: & Retroalimentaci\'on de errores
    en tareas. \\ Alumno: & Resoluci\'on de examen r\'apido (10 incisos) el
    d\'ia 5 de octubre. \\ Profesor: & Retroalimentaci\'on de errores y aciertos
    del examen r\'apido.\\ Alumno: & Correcci\'on de errores y resoluci\'on de
    ejercicios de reforzamiento. \\

\\ \hline \\

    \textsc{Tiempo:}                   & 3 sesiones, 7 -- 9 de octubre. \\
    \textsc{Temas y subtemas:}         & Conversi\'on de fracciones a decimales y viceversa. \\
    \textsc{Aprendizaje esperado: }    & El alumno aprende a convertir n\'umeros
    fraccionarios a decimales y n\'umeros decimales, exactos y preri\'odicos, a
    fraccionarios. \\
    \textsc{Evidencias a evaluar:}      & Dos trabajos y una tarea. \\\\
    \textsc{\large Secuencias:} \\
    Profesor: & Explicaci\'on del procedimiento para convertir n\'umeros
    fraccionarios a decimales. Ejemplos resueltos. Despeje de dudas. \\ Alumno:
    & Resuluci\'on de ejercicios de conversi\'on de n\'umeros fraccionarios a
    decimales. \\ 

\end{tabular}

\begin{tabular}[t]{R{0.25\textwidth}p{0.71\textwidth}}

    Profesor: & Explicaci\'on del procedimiento para convertir
    n\'umeros decimales a fraccionarios. Ejemplos resueltos. Despeje de dudas.
    \\ Alumno: & Resuluci\'on de ejercicios de conversi\'on de n\'umeros
    decimales a fraccionarios. \\ Profesor: & Explicaci\'on del procedimiento
    para convertir n\'umeros decimales peri\'odicos a fraccionarios. Ejemplos
    resueltos. Despeje de dudas. \\ Alumno: & Resuluci\'on de ejercicios de
    conversi\'on de n\'umeros decimales a fraccionarios. Resoluci\'on de una
    tarea con los tres casos anteriores.  \\ Profesor: & Retroalimentaci\'on de
    errores en la tarea.
\\ \hline \\

    \textsc{Tiempo:}                   & 2 sesiones, 12 -- 13 de octubre. \\
    \textsc{Temas y subtemas:}         & Ubicaci\'on de n\'umeros fraccionarios
    y decimales en la recta num\'erica. \\
    \textsc{Aprendizaje esperado: }    & El alumno aprende a ubicar n\'umeros
    fraccionarios y decimales en rectas y circuitos num\'ericos. \\
    \textsc{Evidencias a evaluar:}     & Dos trabajos. \\\\
    \textsc{\large Secuencias:} \\
    Profesor: & Explicaci\'on de la l\'ogica a seguir para ubicar n\'umeros en
    recatas o circuitos num\'ericos: divisi\'on de segmentos. \\ Alumno: &
    Resoluci\'on grupal e individual de ejercicios de rectas num\'ericas.

\\ \hline \\

    \textsc{Tiempo:}                   & 2 sesiones, 14 -- 15 de octubre. \\
    \textsc{Temas y subtemas:}         & Preparaci\'on para examen bimestral. \\
    \textsc{\large Secuencias:} \\
    Ambos: & Sesi\'on de preguntas y respuestas preparativas para el examen bimestral. \\ Alumno: & Resoluci\'on de problemas similares a los que se pide resolver en un examen. \\ Profesor: & Despeje de dudas. 

\\ \hline \\

    \textsc{Tiempo:}& 2 d\'ias, viernes 16 y lunes 19 de octubre. \\
                    & {\Large \sc Aplicaci\'on de examen bimestral} (16 de octubre). \\ 
                    & Revisi\'on grupal del los resultados del examen bimestral (19 de octubre).\\
\\ \hline \\


\end{tabular}



\end{document}
24 ago - 16 oct
Total: 45 sesiones

Sesiones
     (1) 24 agosto: presentación
     (4) 25-27 agosto: primos y minimo conmún múltiplo
     (5) 31-4 septiembre: Criterios de divisibilidad y simplificación fracciones propias
     (4) 7-10 septiembre: mínimo común múltiplo y fraciones equivalentes con mismo denominador
     (1) 11 septiembre: examen rápido de fracciones equivalentes con mismo denominador
     (3) Manejo de fraccines: impropias, mixtas y enteros
     (4) 14-17 septiembre: suma y resta de fracciones con diferente denominador
     (1) 18 septiembre: examen parcial
     (5) 21-25 septiembre: multiplicación y división con fracciones
     (5) 28-2 octubre: jerarquía de las operaciones y operaciones complejas con fracciones y signos de agrupación
     (3) 5-7 octubre: conversión de fracciones a decimales y viceversa
     (2) Problemas que implican operaciones con fracciones
     (3) Problemas que implican operaciones con fracciones
     (2) ubicación de fracciones en la recta numérica 

1+4+5+5+1+4+1+5+5+3+2+5+3

20 + 16 = 36
\section{Informaci\'on  y secuencias did\'acticas}
\begin{tabular}[t]{R{0.25\textwidth}p{0.71\textwidth}}
    \textsc{Tiempo:}                    &1 sesi\'on, 4 de agosto. \\
    \textsc{Temas y subtemas:}          &Presentaci\'on del profesor y de las reglas generales en el salón de clase. \\
    \textsc{Aprendizaje esperado: }     &El alumno retoma y reafirma conocimentos para realizar operaciones basicas con decimales. \\
    \textsc{Evidencias a evaluar:}      &4 tareas y 3 ex\'amenes rapidos. \\\\
    \textsc{\large Secuencias:} \\
    Alumno      &Resoluci\'on de problemas complementarios referentes a operaciones con decimales. \\
    Profesor    &Aplicaci\'on de ex\'amenes r\'apidos para evaluar el el aprendizaje de operaciones con n\'umeros decimales. \\
                &Retroalimentaci\'on personal acerca de las fallas en dichos temas. \\
    Alumno      &Correcci\'on de errores. \\
    Profesor    &Revisi\'on del proceso de aprendizaje. \\ 
\\ \hline \\
\end{tabular}
