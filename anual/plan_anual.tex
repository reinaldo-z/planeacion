\newcommand{\seccion}{SECUNDARIA INCORPORADA A LA SEG }
\newcommand{\descripcion}{Plan anual de matem\'aticas}
\newcommand{\grado}{primero de secundaria}
\newcommand{\ciclo}{Ciclo escolar: 2015--2016}
\newcommand{\papel}{letterpaper} %letter, legalpaper ...

\documentclass[11pt]{article}

\title{\seccion \\ \descripcion, \grado \\ \ciclo}
\newcommand\BackgroundLogo{
\put(142,295){
\parbox[b][\paperheight]{\paperwidth}{%
\vfill
\centering
\includegraphics[width=5cm,height=2.5cm,keepaspectratio]{/Users/reinaldo/Documents/clases/jassa/logo}%
\vfill
}}}

\newcommand{\fecha}{Mi\'ercoles 9 de septiembre de 2015}
\author{M. en C. Reinaldo Arturo Zapata Pe\~na}

\hyphenation{con-ti-nua-ción}

\usepackage{enumitem}
\usepackage[T1]{fontenc} %fuentes
\usepackage{lmodern} %fuente mejorada
\usepackage[spanish]{babel}
\decimalpoint
\usepackage{fullpage}
\usepackage{multicol}
\usepackage{graphicx}
\usepackage{eso-pic}
\usepackage{multirow}
\usepackage{subfigure}
\usepackage[\papel]{geometry}


\usepackage[leqno,fleqn]{amsmath}
\makeatletter
  \def\tagform@#1{\maketag@@@{#1\@@italiccorr}}
\makeatother
\renewcommand{\theequation}{\fbox{\textbf{\arabic{equation}}}}

\begin{document}
\AddToShipoutPicture*{\BackgroundLogo}
\ClearShipoutPicture
\date{\fecha}
\maketitle

\textbf{Prp\'osito:} 

El prop\'osito de las matem\'aticas se centra en propiciar
conocimientos utilizando actividades que despierten el inter\'es de los alumnos
y los inviten a reflexionar para as\'i a encontrar diferentes formas de resolver
los problemas y a formular argumentos que validen los resultados.

\textbf{Competencias que se favorecen:} 

Resolver problemas de manera aut\'onoma.
Comunicar informaci\'on matem\'atica. Validar procedimientos y resultados.
Manejar t\'ecnicas eficientemente.

\textbf{Bibliograf\'ia:} 

Matem\'aticas 1 Cuaderno de Trabajo ED12, Brise\~no Aguirre, L. A., Santillana.

Aritm\'etica te\'orico pr\'actica, Baldor, A., Grupo editorial Patria.

\section*{Bloque 1}

\textbf{Aprendizajes esperados:} al finalizar el primer bloque el alumno aprende a:

\begin{itemize}
\setlength\itemsep{-0.4em}
    \item Identificar n\'umeros primos entre un grupo de n\'umeros.
    \item Descomponer n\'umeros enteros en factores primos.
    \item Usar criterios de divisibilidad.
    \item Calcular el m\'inimo com\'un m\'ultiplo de un grupo de n\'umeros.
    \item Resolver problemas que implican el uso del m\'inimo com\'un m\'ultiplo.
    \item Obtener fracciones equivalentes a partir de una dada.
    \item Convertir un grupo de fracciones equivalentes a fracciones
    equivalentes con com\'un denominador.
    \item Simplificar fracciones hasta su m\'inima expresi\'on.
    \item Convertir fracciones mixtas a impropias y viceversa.
    \item Sumar y restar fraccines con mismo y diferente denominador.
    \item Convertir fracciones mixtas a impropias y viceversa.
    \item Multiplicar y dividir fracciones.
    \item Resolver operaciones combinadas de suma resta multiplicaci\'on y
    divisi\'on acorde a la jerarq\'ia de las operaciones.
    \item Convertir n\'umeros fraccionarios a decimales.
    \item Convertir n\'umeros decimales exactos y peri\'odicos a fraccionarios.
    \item Representar gr\'aficamente fraciones.
    \item Ubiar n\'umeros fraccionarios y decimales en la recta num\'erica.
    \item Resolver problemas que impliquen operaciones de suma y resta de fracciones.
\end{itemize}

\textbf{N\'ucleos matem\'aticos:}

\begin{enumerate}
\setlength\itemsep{-0.4em}
    \item N\'umeros primos.
    \item M\'inimo con\'un m\'ultiplo.
    \item Significado y uso de n\'umeros fraccionarios.
    \item Fracciones equivalentes.
    \item Equivalencia entre fracciones y decimales.
    \item Operaciones con fracciones.
    \item Recta num\'erica
\end{enumerate}

\section*{Bloque 2}

\textbf{Aprendizajes esperados:} al finalizar el primer bloque el alumno aprende a:

\begin{itemize}
\setlength\itemsep{-0.4em}
    \item Resolver operaciones b\'asicas con decimales.
    \item Trazar pol\'igonos regulares circunscritos.
    \item Calcular per\'imetros y \'areas de cuadril\'ateros, tri\'angulos y figuras
    regulares.
    \item Representae sucesiones de n\'umeros o de figuras.
    \item Explicar del significado de f\'ormulas geom\'etricas.
    \item Trazar tri\'angulos y cuadril\'ateros usando el juego de geometr\'ia.
    \item Resolver problemas de reparto proporcional.
    \item Identificar y practicar juegos de azar sencillos y a determinar si un 
    evento tiene probabilidad alta, baja o nula de suceder.
    \item Calcular las medidas de tendencia central: media, moda y mediana.
    \item Resuelvan problemas de conteo con apoyo de representaciones gr\'aficas.
    \item Construir e interpretar histogramas de barras en los que representa
    frecuencias absolutas, relativas y acumuladas.
    \item Resolver problemas aditivos que implican combinaci\'on de n\'umeros
    fraccionarios y decimales.
    \item Trazar bisectrices y mediatrices.
\end{itemize}


\textbf{N\'ucleos matem\'aticos:}
\begin{enumerate}
\setlength\itemsep{-0.4em}
    \item Significado y uso de las operaciones.
    \item Problemas aditivos.
    \item Problemas multiplicativos.
    \item Formas geom\'etricas.
    \item Rectas y \'angulos.
    \item Figuras planas.
    \item Medidas.
    \item Justificaci\'on de f\'ormulas.
    \item An\'alisis de la informaci\'on.
    \item Significado de uso de literales.
    \item Trazo de ejes de simetria, bisectrices y mediatrices.
    \item Proporcionalidad y funciones.
\end{enumerate}

\section*{Bloque 3}

\textbf{Aprendizajes esperados:} al finalizar el primer bloque el alumno aprende a:

\begin{itemize}
\setlength\itemsep{-0.4em}
    \item Resolver ecuaciones lineales de la forma $x+a=b$, $ax=b$, $ax+b=c$.
    \item Resolver problemas que implican ecuaciones lineales.
    \item Resolver problemas que implican el c\'alculo de cualquiera de las
    variables de ls f\'ormulas para calcular per\'imetros y \'areas de figuras.
    \item Calcular per\'imetros y \'areas de figuras compuestas.
    \item Resolver problemas que implican multiplicaci\'on y/o divisi\'on de
    n\'umeros decimales.
    \item Resolverproblemas que implican el c\'alculo de per\'imetros y \'areas
    de pol\'igonos regulares.
    \item Resolver problemas de proporcionalidad directa.
    \item Resolver problemas de porcentajes incluyendo descuentos compuestos.
    \item Interpretar y construir gr\'aficas de pastel.
\end{itemize}

\textbf{N\'ucleos matem\'aticos:}

\begin{enumerate}
\setlength\itemsep{-0.4em}
    \item Problemas multiplicativos.
    \item Ecuaciones.
    \item Figuras planas.
    \item Estimar, medir y calcular.
    \item Relaciones de proporcionalidad.
    \item Porcentajes.
    \item Gr\'aficas

\end{enumerate}

\section*{Bloque4}

\textbf{Aprendizajes esperados:} al finalizar el primer bloque el alumno aprende a:

\begin{itemize}
\setlength\itemsep{-0.4em}
    \item Resolver problemas de proporcionalidad inversa.
    \item Resolver operaciones de n\'umeros con signo.
    \item Plantear y resolver problemas que implican el uso de n\'umeros enteros,
    fraccionarios y decimales, positivos y negativos.
    \item Ubicar puntos en el eje cartesiano.
    \item El proceso de construcci\'on de c\'irculos a partir de diferentes
    dados proporcionados: radio, di\'ametro, cuerda.
    \item Justificar la f\'ormula para calcular la longitud  y el \'area de una
    circunferencia.
    \item Obtener el factor de proporcionalidad.
    \item Obtener los elemenos de una sucesi\'on a partir de una regla dada.
\end{itemize}

\textbf{N\'ucleos matem\'aticos:}

\begin{enumerate}
\setlength\itemsep{-0.4em}
    \item Proporcionalidad inversa.
    \item N\'umeros con signo.
    \item Ubicaci\'on espacial
    \item Figuras y espacio.
    \item Proporcionalidad
    \item Sucesiones y series.
\end{enumerate}


\section*{Bloque 5}

\textbf{Aprendizajes esperados:} al finalizar el primer bloque el alumno aprende a:

\begin{itemize}
\setlength\itemsep{-0.4em}
    \item Jerarquizar las operaciones: orden en el que se deben realizar las
    operaciones.
    \item Resolver operaciones de potenciaci\'on usando n\'umeros enteros,
    decimales y fraccionarios.
    \item Resolver ra\'ices cuadradas exactas usando n\'umeros enteros y
    fraccionarios.
    \item Resolver problemas aditivos que implican n\'umeros con signo.
    \item Resueve problemas que implican interpretar las medidas de tendencia
    central.
\end{itemize}

\textbf{N\'ucleos matem\'aticos:}

\begin{enumerate}
\setlength\itemsep{-0.4em}
    \item N\'umeros con signo.
    \item Potenciaci\'on.
    \item Radicaci\'on.
    \item Justificaci\'on de f\'ormulas.
    \item Estimar, medir y calcular.
    \item Gr\'aficas.
\end{enumerate}


\end{document}






