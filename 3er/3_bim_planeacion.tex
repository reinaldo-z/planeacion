% 7 ene - 26 feb

\documentclass[letterpaper,10pt]{article}
\usepackage[hmargin=3.5cm, vmargin=2.25cm]{geometry} 
\usepackage{color,graphicx}
\usepackage{eso-pic}
\usepackage[usenames,dvipsnames]{xcolor}
% \usepackage{xunicode}
% \usepackage{fontspec}
\usepackage{parskip}
\usepackage[absolute]{textpos}
\usepackage[spanish]{babel}
\spanishdecimal{.}

\usepackage{fullpage}
\usepackage{amsmath}
\newcommand\BackgroundLogo{
\put(175,320){
\parbox[b][\paperheight]{\paperwidth}{%
\vfill
\centering
\includegraphics[width=4cm,height=2cm,keepaspectratio]{/Users/reinaldo/Documents/clases/jassa/logo}
\vfill
}}}

\usepackage{array}
\newcolumntype{L}[1]{>{\raggedright\let\newline\\\arraybackslash\hspace{0pt}}m{#1}}
\newcolumntype{C}[1]{>{\centering\let\newline\\\arraybackslash\hspace{0pt}}m{#1}}
\newcolumntype{R}[1]{>{\raggedleft\let\newline\\\arraybackslash\hspace{0pt}}m{#1}}

%-----------------------------Fonts---------------------------------%
% \defaultfontfeatures{Mapping=tex-text}
% \setmainfont[Mapping=tex-text]{Linux Libertine O}     % http://www.linuxlibertine.org/
% \setsansfont[Scale=MatchLowercase,Mapping=tex-text]{Linux Biolinum O} % http://www.linuxlibertine.org/
% \setmonofont[Scale=MatchLowercase]{FreeMono}          % http://www.gnu.org/software/freefont/

%--------------------Link colors and pdf info-----------------------%
\usepackage{hyperref}
\definecolor{headings}{HTML}{701112}
\definecolor{links}{rgb}{0,0.2,0.6}
\hypersetup{colorlinks,breaklinks,urlcolor=links,linkcolor=links,pdftitle=Planeaci\'on bimestral -- Reinaldo A. Zapata Pe\~na,pdfauthor=Reinaldo A. Zapata Pe\~na}

%-----------------------------Watermark-----------------------------%
\setlength{\TPHorizModule}{30mm}
\setlength{\TPVertModule}{\TPHorizModule}
\textblockorigin{4mm}{0.2\paperheight}
\setlength{\parindent}{0pt}

%----------------------Custom section headings----------------------%
\usepackage{titlesec}
\titleformat{\section}
{\color{headings}\scshape\sffamily\Large\raggedright}{}{0em}{}[\color{black}\titlerule]
\titlespacing{\section}{0pt}{3pt}{3pt}

\title{\sc \Huge Secuencia Did\'actica Bimestral \\ Primero de Secundaria}


%-------------------------------------------------------------------%
%-----------------------BEGINNING OF DOCUMENT-----------------------%
%-------------------------------------------------------------------%

\begin{document}
\AddToShipoutPicture*{\BackgroundLogo}
\ClearShipoutPicture

\date{}

\maketitle

\vspace{-2.3cm}

\begin{center}
{\Large\color{headings}\sffamily{Secundaria Incorporada a la SEP  }}
\end{center}

% \vspace{5mm}

\section{Informaci\'on general}
\begin{tabular}{R{0.25\textwidth}p{0.71\textwidth}}
    \textsc{Docente:}       &   M. en C. Reinaldo Arturo Zapata Pe\~na. \\
    \textsc{Materia:}       &   Matem\'aticas, primero de secundaria. \\
    \textsc{Grupos:}        &   A, C. \\
    \textsc{Ciclo escolar:} &   2015 -- 2016. \\
    \textsc{Periodo:}       &   Tercer bimestre, 7 de enero -- 26 de febrero de 2016. \\
                            &   Un total de 42 sesiones de 50 minutos. \\
    \textsc{Criterios de}   &   Examen parcial:     35\%. \\   
    \textsc{evaluaci\'on:}  &   Examen bimestral:   35\%. \\
                            &   Tareas y trabajos:  25\%. \\
                            &   Participaci\'on:     5\%. \\
    \textsc{Seguimiento:}   &   Es necesario que los alumnos obtengan un
    promedio de examen aprobatorio para que sus trabajos en clase y la
    participación sean tomados en cuenta. Se mandar\'an avisos a casa para los
    padres de aquellos alumnos que no entreguen dos o m\'as tareas. \\
    & Los siguientes alumnos ser\'an evaluados constantemente en el \'area de operaciones b\'asicas con fracciones y decimales a manera de mejorar su desempe\~no en clases: \\
    & 1A: Avila Ruiz Evelyn Daniela, Cordero López Eugenio, Cornejo Tovar Axel Ariel, Lopez Bojorges Mariana, Ocampo Camacho Maria Jose, Villegas Perez Maximiliano. \\
    & 1B: De Armero Hinojosa Renata Sofía, Dominguez Villaseñor Juan Pablo, Ehnis Borja Rodrigo Eugenio, Gonzalez Paz Nathalia Nicole, Landeros Rocha María José, Marquez Amaro Leigh Ann, Martinez Juarez Juan Miguel, Pérez Garrido Emilio, Ventura Estrada Eric Josue, Zamora Arellano Karla Daniela.

\end{tabular}


\section{Competencias a trabajar}
\begin{tabular}[t]{R{0.25\textwidth}p{0.71\textwidth}}
    \textsc{Competencias: } &   Resolver problemas de manera aut\'onoma. \\ 
                            &   Comunicar información matem\'atica. \\
                            &   Validar procedimientos y resultados. \\
                            &   Manejar t\'ecnicas eficientemente. \\
\end{tabular}

% \section{Resumen de actividades}
% \begin{tabular}[t]{R{0.25\textwidth}p{0.71\textwidth}}
% 16 octubre      & Dictado temario, portada segundo bimestre. \\
% 19-23 octubre   & Revision examen bimestral, Per\'imetrosy 
%                   \'areas de figuras regulares. \\
% 26-29 octubre   & Per\'imetros y \'areas de figuras regulares e irregulares \\
% 30 octubre      & Consejo T\'ecnico Escolar \\
% 3-6, 9-11 noviembre   & Operaciones y problemas con decimales. \\
% 12-13, 16-19 noviembre & Figuras irregulares y problemas con per\'imetros y \'areas. \\
% 20 noviembre    & Examen parcial. \\
% 23 noviembre    & Revisi\'on de examen, correcci\'on de examen parcial. \\
% 24-26 noviembre & Proporcionalidad directa. \\
% 27 noviembre    & Consejo T\'ecnico escolar. \\
% 30-3 diciembre  & Proporcionalidad directa e inversa. \\
% 7-11 diciembre  & Medidas de tendencia central / Probabilidad y estad\'istica. \\
% 14 diciembre    & Examen bimestral. \\
% \end{tabular}

\newpage

\section{Aprendizajes esperados}
\begin{tabular}[t]{R{0.25\textwidth}p{0.71\textwidth}}
% Fracciones:     & Los alumnos reafirma sus conocimientos en el manejo de
% fracciones, operaciones b\'asicas con fracciones y problemas con fracciones
% similares a los trabajados en el primer bimestre. \\ 
% Decimales:      & Losalumnos aprenden a resolver sumas y restas con decimales
%                 acomodando en punto decimal en la forma correcta y recordando el
%                 proceso correspondiente para cada uno de los procesos. \\
%                 & Los alumnos aprenden a resolver multiplicaciones con decimales
%                 acomodando el punto decimal en el resutado final. \\
%                 & Los alumnos aprenden a resolver divisiones con decimales en el
%                 divdendo y/o divisor. \\
%                 & Los alumnos aprenden a resolver problemas que implican el uso
%                 de operaciones con decimales. \\
Proporcionalidad: 
                & Los alumnos aprenden a resolver problemas de
                proporcionalidad as\'i como porcentajes sencillos como 25\%, 50\%
                y 75\% y y descuentos porcentuales complejos. \\
Geometr\'ia:     
                Los alumnos reafirman sus conocimientos en el trazo de mediatrices
                y bisectrices y los aplican para resolver problemas.
                & Usando fracciones y decimales los alumnos reafirman sus
                conocimientos en el c\'alculo de  
                per\'imetros y el \'areas de figuras planas regulares e 
                irregulares. \\
                & Los alumnos aprenden a resolver prblemas que implican el
                c\'alculo de per\'imetros y \'areas de figuras planas. \\
Probabilidad y estad\'istica:
                & Los alumnos aprenden a calcular las medidas de tendencia 
                central (media, moda y mediana) de grupos de datos. \\
MCM y MCD: &    Los alumnos aprenden a resolver problemas que implican el 
                c\'alculo del m\'inimo com\'un m\'ultiplo (MCM) y el m\'aximo 
                com\'un divisor (MCD). \\
\'Algebra: &    El alumno se familiariza con las expresiones 
                algebraicas reconociendo sus partes: coeficiente, literal y
                exponente. El alumno aprende a resolver ecuaciones lineales con
                una sola variable y lo aplica a problemas.
\end{tabular}

\vspace{5mm}

El enfoque de la clase se har\'a impulsndo tanto el trabajo individual como el 
grupal fomentando la sana convivencia entre los alumnos as\'i como el esp\'iritu
de ayuda mutua para lograr un avance en los conocimientos dando lugar a un 
aprendizaje colaborativo mediante la ayuda/tutor\'ia entre pares.

\vspace{5mm}

\section{Temas  y secuencias did\'acticas}
\begin{tabular}[t]{R{0.25\textwidth}p{0.71\textwidth}}
    \textsc{Tiempo:}                    & 2 sesiones: 7-8 de enero. \\
    \textsc{Temas y subtemas:}          & Problemas que implican el uso de multiplicaciones y divisiones con fracciones.\\
    \sc{Evidencias a evaluar:}          & Una tarea correspondiente a las 
    p\'aginas 64-67 del libro. \\ \\
    \large{\sc Secuencias:} \\
    Profesor:   & Se les har\'a un recordatorio del procedimiento para resolver 
    multiplicaciones y divisiones con fracciones. \\
    Alumno:     & Leer atentamente la lecci\'on de la p\'agina 64 y resolver los 
    ejercicios/problemas correspondientes a las p\'aginas 65-67 del libro. \\
    Profesor: & Explicación detallada del procedimiento para resolver la \'ultima
    secci\'on de problemas de la p\'agina 67 haciendo trabajo junto con los 
    alumnos.

\end{tabular}
\begin{tabular}[t]{R{0.25\textwidth}p{0.71\textwidth}}

\\ \hline \\
    \textsc{Tiempo:}                    & 5 sesiones: 4--8  de enero. \\
    \textsc{Temas y subtemas:}          & Proporcionalidad, regla de tres y porcentajes.\\
    \sc{Evidencias a evaluar:}          & 2 tareas, 4 ejercicio en libreta\\ \\
    \large{\sc Secuencias:} \\
    Alumno:     & Obtenci\'on de la definici\'on de per\'imetro y \'area usando el diccionario. \\
                & Diferenciaci\'on entre per\'imetro y \'area y sus correspondientes unidades \\
    Profesor:   & Dictado de las f\'ormulas para obtener el per\'imetro y el \'area de figuras planas regulares. \\
                & Explicaci\'on del uso de la jerarqu\'ia de operaciones para resolver el \'area de una circunferencia y de la procedencia de la misma: $A_{c} = \pi r^{2}$. \\ 
                & Explicaci\'on de la procedencia del \'area del tri\'angulo: $A_{t} = \frac{bh}{2}$. \\
                & Explicaci\'on del trazo de tri\'angulos equil\'ateros. \\
    Alumno:     & Construcci\'on de circunferencias, tri\'angulos equil\'ateros y cuadrads y c\'alculos de sus correspondientes per\'imetros y \'areas. \\
    Profesor:   & Explicaci\'on de la procedencia para el c\'alculo de per\'imetros y \'aras de pol\'igonos regulares con cinco o m\'as lados: $P = n\ell$, $A = \frac{Pa}{2} $. \\
    Alumno:     & C\'alculo del per\'imetro y \'area de pol\'igonos regulares de cinco o m\'as lados. \\
\\ \hline \\
    \textsc{Tiempo:}& 1 d\'ia, viernes 30 de Octubre. \\
                    & {\Large \sc Consejo T\'ecnico Escolar.} \\ 
\\ \hline \\
    \textsc{Tiempo:}                    & 7 sesiones: 3--6 y 9--11  de noviembre. \\
    \textsc{Temas y subtemas:}          & Operaciones con n\'umeros decimales.\\
    \sc{Evidencias a evaluar:}          & 2 tarea, 3 ejercicios en libreta, p\'aginas del libro (60--63, 88--95) y examen sorpresa. \\ \\
    \large{\sc Secuencias:} \\
    Pofesor:    & Revisi\'on con los alumnos de procedimientos incorrectos al momento de hacer sumas y restas con n\'umeros enteros y decimales. \\
    Alumno:     & Correcci\'on de procedimientos al momento de acomodar el punto decimal, y al momento de restar enteros con decimales.\\
    Alumno:     & Resoluci\'on de operaciones de suma y resta con decimales. Resolución de p\'agimas del libro: 60--63 \\
    Profesor:   & Explicaci\'on del procedimiento correcto para resolver multiplicaciones con decimales: conteo de posiciones decimales en los factores. \\

    Alumno:     & Resoluci\'on de multiplicaciones con decimales. \\
                & Resoluci\'on de p\'aginas del libro: 88--91 \\
    Profesor:   & Explicaci\'on del procedimento para resolver correctamente divisiones con decimales: recoriendo el punto decimal de forma correcta. \\
    Alumno:     & Resoluci\'on de divisiones con punto decimal. \\ 
                & Resoluci\'on de p\'aginas del libro: 92--95. \\ \\
    Alumno:     & Resoluci\'on de un examen sorpresa con valor de una tarea.
\\ \hline \\
\end{tabular}

\begin{tabular}[t]{R{0.25\textwidth}p{0.71\textwidth}}

    \textsc{Tiempo:}                    & 6 sesiones: 12--13 y 16--19  de noviembre. \\
    \textsc{Temas y subtemas:}          & Per\'imetros y \'areas de figuras planas iregulares y problemas que implican per\'imetros y \'areas de figuras planas.\\
    \sc{Evidencias a evaluar:}          & 1 tarea, 2 ejercicio en libreta y p\'agins del libro (36-39, 76--79) \\ \\
    \large{\sc Secuencias:} \\
    Profesor:   & Dictado, escritura y explicaci\'on de las f\'ormulas para calcular \'areas de figuras irregulares. \\
    Alumno:     & C\'alculo de per\'imetros y \'areas de figuras irregulares. \\
                & Medici\'on de los par\'ametros de las figuras de una cancha de b\'asquetbol.  \\
                & Resoluci\'on de las p\'aginas del libro correspondientes a este tema: 24--31, 104--107. \\
    Profesor:   & Planteamiento de la actividad \textit{Jugando al arquitecto}: en esta actividad se le plantea al alumno el problema de construir una plaza p\'ublica que implica distintas figuras geom\'etricas; posteriormente sele pide que plantee que materiales se necesitar\'an y que saque un costo final del proyecto. \\
    Alumno:     & Resoluci\'on del proyecto de construcci\'on de una plaza p\'ublica. C\'alculo de costos finales. C\'alculo de ganancias.\\
\\ \hline \\
 
    \textsc{Tiempo:}& 2 sesiones: lunes 20 y martes 21  de noviembre. \\
                    & {\Large \sc Aplicaci\'on de examen parcial} (20 de noviembre). \\ 
                    & Revisi\'on grupal del los resultados del examen parciall y correcci\'on de examen (22 de noviembre).\\
\\ \hline \\
    \textsc{Tiempo:}                    & 6 sesiones: 24--26, 30 de noviembre y 1--3 de diciembre \\
    \textsc{Temas y subtemas:}          & Proporcionalidad directa e inversa. \\
    \sc{Evidencias a evaluar:}          & 1 tarea, 2 ejercicio en libreta, p\'agins del libro (24--31) y examen sorpresa \\ \\

    \large{\sc Secuencias:} \\
    Profesor:   & Explicaci\'on de la importancia de la proporcionalidad. \\
                & Explicaci\'on del procedimiento para resolver regla de tres. \\
    Alumno:     & Resolución de problemas de que implican el uso de regla de tres. \\
                & Resoluci\'on de problemas en el libro: p\'aginas 24--31. \\
    Profesor:   & Explicaci\'on de porcentajes a manera de regla de tres. \\
                & Explicaci\'on de estrategia mercantil: descuento sobre descuento \\
                & Exlicaci\'on de porcentajes notables. \\
    Alumno:     & C\'alculo de porcentajes sencillos y compuestos. \\
    Profesor:   & Explicación de la diferencia entre la proporcionalidad directa y la inversa. Ejemplos de problemas. \\
    Alumno:     & Resoluci\'on de problemas que implican el uso de proporcionalidad inversa. \\
\\ \hline \\
\end{tabular}

\begin{tabular}[t]{R{0.25\textwidth}p{0.71\textwidth}}
\\ \hline \\
    \textsc{Tiempo:}& 1 d\'ia, viernes 30 de Octubre. \\
                    & {\Large \sc Consejo T\'ecnico Escolar.} \\ 
\\ \hline \\
    \textsc{Tiempo:}                    & 5 sesiones: 7--11 de diciembre. \\
    \textsc{Temas y subtemas:}          & Medidas de tendencia central: media, moda y mediana; Nociones de probabilidad.\\
    \sc{Evidencias a evaluar:}          & 1 tarea, 2 ejercicio en libreta y p\'agins del libro (40--43) \\ \\
    \large{\sc Secuencias:} \\
    Profesor:   & Explicaci\'on de la importancia de las medidas de tendencia central y su uso: media, moda y mediana. \\
                & Explicaci\'on del procedimiento para la obtenci\'on de dichas medidas de tendencia central. \\
    Alumno:     & C\'alculo de las medidas de tendencia central de un juego de n\'umeros.  \\
                & C\'alculo de las medidas de tendencia central de distintos datos: edades, alturas, distancias. \\
    Profesor:   & Explicaci\'on de la probabilidad de un evento. \\
                & Explicaci\'on de c\'omo es que funcionan algunos juegos de azar. \\
    Alumno:     & Determinación de la probabilidad de distintos eventos: alta, media, baja o nula. \\
    Profesor:   & Explicaci\'on de la construcci\'on de un histogramas y de c\'omo extraer datos de los mismos. \\
    Alumno:     & Consrtrucci\'on de histogramas a partir de datos estad\'isticos.\\
\\ \hline \\
 
    \textsc{Tiempo:}& 2 sesiones: lunes 14 y martes 15  de diciembre. \\
                    & {\Large \sc Aplicaci\'on de examen parcial} (14 de diciembre). \\ 
                    & Revisi\'on grupal del los resultados del examen parciall y correcci\'on de examen (15 de diciembre).\\
\\ \hline \\

\end{tabular}


\end{document}



16 Octo - 17 Dic
Total: 42 sesiones

---- temas parcial
16 octubre      1 Dictado temario, portada segundo bimestre
19-23 octubre   5 Revision examen bimestral, corrección examen, Perimetros/Areas fig. reg
26-29 octubre   4 Perímetros y áreas de figuras regulares
30 octubre CTE  0 Consejo Técnico Escolar
3-6 noviembre   4 Operaciones decimales
9-11 noviembre  7 Operaciones decimales y problemas
12-13 16-19 noviembre 6 Problemas perímetros y áreas
20 noviembre    1 Examen parcial
---- temas bimestral
23 noviembre    1 Revisión de examen, corrección de examen
24-26 noviembre 3 Proporcionalidad directa
27 nov CTE      0 
30-3 diciembre  5 Proporcionalidad directa e inversa
7-11 diciembre  5 Medidas de tendencia central / Probabilidad y estadística
14-16 diciembre 3 Histogramas / Fonomímicas / 
17 diciembre    1 Examen bimestral
        Total: 42

\section{Informaci\'on  y secuencias did\'acticas}
\begin{tabular}[t]{R{0.25\textwidth}p{0.71\textwidth}}
    \textsc{Tiempo:}                    &1 sesi\'on: 4 de agosto. \\
    \textsc{Temas y subtemas:}          &Presentaci\'on del profesor y de las reglas generales en el salón de clase. \\
    \textsc{Aprendizaje esperado: }     &El alumno retoma y reafirma conocimentos para realizar operaciones basicas con decimales. \\
    \textsc{Evidencias a evaluar:}      &4 tareas y 3 ex\'amenes rapidos. \\\\
    \textsc{\large Secuencias:} \\
    Alumno      &Resoluci\'on de problemas complementarios referentes a operaciones con decimales. \\
    Profesor    &Aplicaci\'on de ex\'amenes r\'apidos para evaluar el el aprendizaje de operaciones con n\'umeros decimales. \\
                &Retroalimentaci\'on personal acerca de las fallas en dichos temas. \\
    Alumno      &Correcci\'on de errores. \\
    Profesor    &Revisi\'on del proceso de aprendizaje. \\ 
\\ \hline \\
\end{tabular}
