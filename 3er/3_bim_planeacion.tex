\documentclass[letterpaper,10pt]{article}
\usepackage[hmargin=3.5cm, vmargin=2.25cm]{geometry} 
\usepackage{color,graphicx}
\usepackage{eso-pic}
\usepackage[usenames,dvipsnames]{xcolor}
% \usepackage{xunicode}
% \usepackage{fontspec}
\usepackage{parskip}
\usepackage[absolute]{textpos}
\usepackage[spanish]{babel}
\spanishdecimal{.}

\usepackage{fullpage}
\usepackage{amsmath}
\newcommand\BackgroundLogo{
\put(175,320){
\parbox[b][\paperheight]{\paperwidth}{%
\vfill
\centering
\includegraphics[width=4cm,height=2cm,keepaspectratio]{/Users/reinaldo/Documents/clases/jassa/logo}
\vfill
}}}

\usepackage{array}
\newcolumntype{L}[1]{>{\raggedright\let\newline\\\arraybackslash\hspace{0pt}}m{#1}}
\newcolumntype{C}[1]{>{\centering\let\newline\\\arraybackslash\hspace{0pt}}m{#1}}
\newcolumntype{R}[1]{>{\raggedleft\let\newline\\\arraybackslash\hspace{0pt}}m{#1}}

%-----------------------------Fonts---------------------------------%
% \defaultfontfeatures{Mapping=tex-text}
% \setmainfont[Mapping=tex-text]{Linux Libertine O}     % http://www.linuxlibertine.org/
% \setsansfont[Scale=MatchLowercase,Mapping=tex-text]{Linux Biolinum O} % http://www.linuxlibertine.org/
% \setmonofont[Scale=MatchLowercase]{FreeMono}          % http://www.gnu.org/software/freefont/

%--------------------Link colors and pdf info-----------------------%
\usepackage{hyperref}
\definecolor{headings}{HTML}{701112}
\definecolor{links}{rgb}{0,0.2,0.6}
\hypersetup{colorlinks,breaklinks,urlcolor=links,linkcolor=links,pdftitle=Planeaci\'on bimestral -- Reinaldo A. Zapata Pe\~na,pdfauthor=Reinaldo A. Zapata Pe\~na}

%-----------------------------Watermark-----------------------------%
\setlength{\TPHorizModule}{30mm}
\setlength{\TPVertModule}{\TPHorizModule}
\textblockorigin{4mm}{0.2\paperheight}
\setlength{\parindent}{0pt}

%----------------------Custom section headings----------------------%
\usepackage{titlesec}
\titleformat{\section}
{\color{headings}\scshape\sffamily\Large\raggedright}{}{0em}{}[\color{black}\titlerule]
\titlespacing{\section}{0pt}{3pt}{3pt}

\title{\sc \Huge Secuencia Did\'actica Bimestral \\ Primero de Secundaria}
\author{Reinaldo Arturo Zapata Pe\~na}

%-------------------------------------------------------------------%
%-----------------------BEGINNING OF DOCUMENT-----------------------%
%-------------------------------------------------------------------%

\begin{document}
\AddToShipoutPicture*{\BackgroundLogo}
\ClearShipoutPicture

\date{}

\maketitle

\vspace{-2.3cm}

\begin{center}
{\Large\color{headings}\sffamily{Secundaria Incorporada a la SEP  }}
\end{center}

% \vspace{5mm}

\vfill

\section{Informaci\'on general}
\begin{tabular}{R{0.25\textwidth}p{0.71\textwidth}}
    \textsc{Docente:}       &   M. en C. Reinaldo Arturo Zapata Pe\~na. \\
    \textsc{Materia:}       &   Matem\'aticas, primero de secundaria. \\
    \textsc{Grupos:}        &   A, C. \\
    \textsc{Ciclo escolar:} &   2015 -- 2016. \\
    \textsc{Periodo:}       &   Tercer bimestre, 7 de enero -- 26 de febrero de 2016. \\
                            &   Un total de 42 sesiones de 50 minutos. \\
    \textsc{Criterios de}   &   Examen parcial:     35\%. \\   
    \textsc{evaluaci\'on:}  &   Examen bimestral:   35\%. \\
                            &   Tareas y trabajos:  25\%. \\
                            &   Participaci\'on:     5\%. \\
    \textsc{Seguimiento:}   &   Es necesario que los alumnos obtengan un
    promedio de examen aprobatorio para que sus trabajos en clase y la
    participación sean tomados en cuenta. Se mandar\'an avisos a casa para los
    padres de aquellos alumnos que no entreguen dos o m\'as tareas. \\ & Los
    siguientes alumnos ser\'an evaluados constantemente en el \'area de
    operaciones b\'asicas con fracciones y decimales a manera de mejorar su
    desempe\~no en clases: \\
    & 1A: Avila Ruiz Evelyn Daniela, Cordero López Eugenio, Cornejo Tovar Axel
    Ariel, Lopez Bojorges Mariana, Ocampo Camacho Maria Jose, Villegas Perez
    Maximiliano. \\    
    & 1B: De Armero Hinojosa Renata Sofía, Dominguez Villaseñor Juan Pablo,
    Ehnis Borja Rodrigo Eugenio, Gonzalez Paz Nathalia Nicole, Landeros Rocha
    María José, Marquez Amaro Leigh Ann, Martinez Juarez Juan Miguel, Pérez
    Garrido Emilio, Ventura Estrada Eric Josue, Zamora Arellano Karla Daniela.

\end{tabular}

\vfill

\section{Competencias a trabajar}
\begin{tabular}[t]{R{0.25\textwidth}p{0.71\textwidth}}
    \textsc{Competencias: } &   Resolver problemas de manera aut\'onoma. \\ 
                            &   Comunicar información matem\'atica. \\
                            &   Validar procedimientos y resultados. \\
                            &   Manejar t\'ecnicas eficientemente. \\
\end{tabular}

% \section{Resumen de actividades}
% \begin{tabular}[t]{R{0.25\textwidth}p{0.71\textwidth}}
% 16 octubre      & Dictado temario, portada segundo bimestre. \\
% 19-23 octubre   & Revision examen bimestral, Per\'imetrosy 
%                   \'areas de figuras regulares. \\
% 26-29 octubre   & Per\'imetros y \'areas de figuras regulares e irregulares \\
% 30 octubre      & Consejo T\'ecnico Escolar \\
% 3-6, 9-11 noviembre   & Operaciones y problemas con decimales. \\
% 12-13, 16-19 noviembre & Figuras irregulares y problemas con per\'imetros y \'areas. \\
% 20 noviembre    & Examen parcial. \\
% 23 noviembre    & Revisi\'on de examen, correcci\'on de examen parcial. \\
% 24-26 noviembre & Proporcionalidad directa. \\
% 27 noviembre    & Consejo T\'ecnico escolar. \\
% 30-3 diciembre  & Proporcionalidad directa e inversa. \\
% 7-11 diciembre  & Medidas de tendencia central / Probabilidad y estad\'istica. \\
% 14 diciembre    & Examen bimestral. \\
% \end{tabular}

\newpage

\section{Aprendizajes esperados}
\begin{tabular}[t]{R{0.25\textwidth}p{0.71\textwidth}}
% Fracciones:     & Los alumnos reafirma sus conocimientos en el manejo de
% fracciones, operaciones b\'asicas con fracciones y problemas con fracciones
% similares a los trabajados en el primer bimestre. \\ 
% Decimales:      & Losalumnos aprenden a resolver sumas y restas con decimales
%                 acomodando en punto decimal en la forma correcta y recordando el
%                 proceso correspondiente para cada uno de los procesos. \\
%                 & Los alumnos aprenden a resolver multiplicaciones con decimales
%                 acomodando el punto decimal en el resutado final. \\
%                 & Los alumnos aprenden a resolver divisiones con decimales en el
%                 divdendo y/o divisor. \\
%                 & Los alumnos aprenden a resolver problemas que implican el uso
%                 de operaciones con decimales. \\
Proporcionalidad: 
                & Los alumnos aprenden a resolver problemas de
                proporcionalidad as\'i como porcentajes sencillos como 25\%, 50\%
                y 75\% y y descuentos porcentuales complejos. \\

Geometr\'ia:    & Los alumnos reafirman sus conocimientos en el trazo de mediatrices
                y bisectrices y los aplican para resolver problemas. \\
                & Usando fracciones y decimales los alumnos reafirman sus
                conocimientos en el c\'alculo de  
                per\'imetros y el \'areas de figuras planas regulares e 
                irregulares. \\
                & Los alumnos aprenden a resolver prblemas que implican el
                c\'alculo de per\'imetros y \'areas de figuras planas. \\
Probabilidad y estad\'istica:
                & Los alumnos aprenden a calcular las medidas de tendencia 
                central (media, moda y mediana) de grupos de datos. \\
MCM y MCD: &    Los alumnos aprenden a resolver problemas que implican el 
                c\'alculo del m\'inimo com\'un m\'ultiplo (MCM) y el m\'aximo 
                com\'un divisor (MCD). \\
\'Algebra: &    El alumno se familiariza con las expresiones 
                algebraicas reconociendo sus partes: coeficiente, literal y
                exponente. El alumno aprende a hacer supresi\'on de t\'erminos
                semejantes lineales. El alumno aprende a resolver ecuaciones
                lineales con una sola variable y lo aplica a problemas.
\end{tabular}


\section{Enfoque y metodolog\'ia}

El enfoque de la clase se har\'a impulsando tanto el trabajo individual como el 
grupal fomentando la sana convivencia entre los alumnos as\'i como el esp\'iritu
de ayuda mutua para lograr un avance en los conocimientos dando lugar a un 
aprendizaje colaborativo mediante la ayuda/tutor\'ia entre pares.


\section{Temas  y secuencias did\'acticas}
\begin{tabular}[t]{R{0.25\textwidth}p{0.71\textwidth}}
    \textsc{Tiempo:}          & 2 sesiones: 7-8 de enero. \\
    \textsc{Temas y subtemas:}& Problemas que implican el uso de 
    multiplicaciones y divisiones con fracciones.\\
    \sc{Evidencias a evaluar:}& Un trabajo correspondiente a las 
    p\'aginas 64-67 del libro. \\ \\
    \large{\sc Secuencias:} \\
    Profesor:   & Se les har\'a un recordatorio del procedimiento para resolver 
    multiplicaciones y divisiones con fracciones. \\
    Alumnos:     & Leer atentamente la lecci\'on de la p\'agina 64 y resolver los 
    ejercicios/problemas correspondientes a las p\'aginas 65-67 del libro. \\
    Profesor: & Explicación detallada del procedimiento para resolver la \'ultima
    secci\'on de problemas de la p\'agina 67 haciendo trabajo junto con los 
    alumnos. \\ 
\hline
\end{tabular}
\\
\newpage
\section{Temas  y secuencias did\'acticas (continuaci\'on)}

\begin{tabular}[t]{R{0.25\textwidth}p{0.71\textwidth}}
    \textsc{Tiempo:}          & 5 sesiones: 4--8  de enero. \\
    \textsc{Temas y subtemas:}& Proporcionalidad, regla de tres y porcentajes.\\
    \sc{Evidencias a evaluar:}& 2 tareas, 4 ejercicio en libreta\\ \\
    \large{\sc Secuencias:} \\
    
    Profesor:   & Recapitulaci\'on del proceso para resolver problemas de
    proporcionalidad. Dictado de problemas ejemplo y resoluci\'on. Dictado de
    problemas de proporcionalidad.\\    
    Alumnos:     & Resoluci\'on de problemas de proporcionalidad, recordando
    as\'i la metodolog\'ia trabajada previamente.  \\
    Profesor:   & Explicaci\'on de problemas porcentuales como un caso especial
    a los problemas de proporcionalidad. \\    
    Alumnos:     & Resoluci\'on de problemas de porcentajes simples: tanto por
    ciento y descuentos. \\    
    Profesor:   & Explicaci\'on del procedimiento para resolver problemas de
    descuentos compuestos tales como ``50\% + 20\% de descuento''. Explicaci\'on
    del por qu\'e el descuento total no es la suma de los descuentos.
    Planteamiento del problema ``Venta nocturna'': 50\% + 20\% de descuento +
    20\% en monedero electr\'onico en art\'iculos seleccionados y 18
    mensualidades sin intereses. \\ Alumnos:     & Selecciona que art\'iculos
    quiere comprar en la venta nocturna y hace el c\'alculo de los precios
    finales, cantidad obtenida en monedero y cantidad mensual a pagar.
\\ \hline \\
    \textsc{Tiempo:}           & 5 sesiones: 18--22  de enero. \\
    \textsc{Temas y subtemas:} & bisectriz y mediatriz, per\'imetros y \'areas
    y proporcionalidad.\\
    \sc{Evidencias a evaluar:} & Lecciones 14, 15, 16 y evaluaci\'on 
    correspondientes a las p\'aginas 68-71, 72-75, 76-79, y 80-81, 
    respectivamente.\\ 
    \large{\sc Secuencias:} \\

    Alumnos:    & Lectura de comprensi\'on de la explicaci\'on de cada una de
    las lecciones: primera p\'agina de cada lecci\'on. Resoluci\'on de los
    ejercicios correspondientes de cada lecci\'on: segunda, tercera y cuarta
    p\'agina de cada lecci\'on.  \\

    Profesor:   & Resoluci\'on de dudas individuales de los ejercicios
    correspondientes. Ayuda grupal con ejercicios mas complicados. \\

    Alumnos:     & Resolici\'on dela evaluaci\'on. \\
\hline\\

    \textsc{Tiempo:}          & 4 sesiones: 25--28  de enero. \\
    \textsc{Temas y subtemas:}& Introducci\'on al \'algebra:
    identificaci\'on de las partes de una expresi\'on algebraica; supresi\'on de
    t\'erminos semejantes de expresiones algebraicas lineales.\\    
    \sc{Evidencias a evaluar:}& 1 tareas, 2 ejercicio en libreta\\ \\
    \large{\sc Secuencias:} \\
    Profesor:   & Explicaci\'on de la representaci\'on de n\'umeros mediante
    letras: casos cotidianos tales como las expresiones usadas para calcular
    per\'imetros y \'areas. \\ & Explicaci\'on de las partes que componen a un
    expresi\'on algebraica: coeficiente, literal(es) y exponente(s). \\    

    Alumnos:     & Tabla de identificaci\'on de las partes que componen a un
    expresi\'on algebraica en distintas expresiones. \\    
    
\end{tabular}

\newpage
\section{Temas  y secuencias did\'acticas (continuaci\'on)}

\begin{tabular}[t]{R{0.25\textwidth}p{0.71\textwidth}}

    Profesor:   & Explicaci\'on con \emph{peras y manzanas} del procedimiento
    para hacer supresi\'on de t\'erminos semejantes: las peras son representadas
    por la letra \emph{p} y las manzanas por la letra \emph{m}. \\    
    Alumnos:     & Reducci\'on de t\'erminos semejantes de expresiones similares
    a $3m+2p-m+4p=2m+6p$. \\    
    Profesor:   & Paso de lo concreto (\emph{peras y manzanas}) a lo abstracto
    usando otras variables: $a,b,c,x,y,z$. \\    
    Alumnos:     & Reducci\'on de t\'erminos semejantes usando expresiones
    algebraicas con distintas literales.
\\ \hline \\
    \textsc{Tiempo:}& 1 d\'ia, viernes 29 de enero. \\
                    & {\Large \sc Consejo T\'ecnico Escolar.} \\ 
\\ \hline \\
\textsc{Tiempo:}          & 2 sesiones: 2--3  de febrero. \\
    \textsc{Temas y subtemas:}& Geometr\'ia: peri\'imetros y \'areas.\\    
    \sc{Evidencias a evaluar:}& 1 trabajo en libreta, lecci\'on 21 del libro 
    (p\'ags. 104-107). \\ \\
    \large{\sc Secuencias:} \\
    Profesor:   & Planteamiento del problema de construcci\'on de una fuente y
    dise\~no de un jard\'in circundante. \\
    Alumnos y profesor:& Sugerencias de los elementos que se emplear\'an para 
    construir la fuente y el jard\'in circundante. \\
    Alumnos:    & Uso del c\'alculo de per\'imetros y \'areas y de la regla de
    tres para la obtenci\'on de los precios finales. Obtenci\'on del costo 
    final.
\\ \hline \\
\textsc{Tiempo:}          & 2 sesiones: 4--5  de febrero. \\
    \textsc{Temas y subtemas:}& Repaso general para examen parcial.\\    
    \sc{Evidencias a evaluar:}& 2 trabajos en libreta, 2 tareas y lecci\'on 22 
    del libro (p\'ags 108-111). \\ \\
    \large{\sc Secuencias:} \\
    Profesor:   & Recapitulaci\'on de los temas vistos a lo largo de la primera 
    mitad del bimestre en curso y de algunos temas de importancia vistos a lo 
    largo del ciclo escolar. \\
    Alumnos     & Resoluci\'on de ejercicios similares a los correspondientes al
    examen parcial. \\
    Profesor:   & Resoluci\'on de dudas.
    \textsc{Tiempo:}& 2 sesiones: lunes 8 y martes 9  de febrero. \\
            & {\Large \sc Aplicaci\'on de examen parcial} (8 de febrero). \\           
            & Revisi\'on grupal del los resultados del examen parcial y
            correcci\'on de examen (9 de febrero).\\
\\ \hline \\
\end{tabular}

\newpage

\section{Temas  y secuencias did\'acticas (continuaci\'on)}

\begin{tabular}[t]{R{0.25\textwidth}p{0.71\textwidth}}
    \textsc{Tiempo:}           & 3 sesiones: 10--13  de febrero. \\
    \textsc{Temas y subtemas:} & Probabilidad y estad\'istica: medidas de
    tendencia central \\
    \sc{Evidencias a evaluar:} & 3 trabajos en libreta, 1 tarea y lecciones 23 
    y 24 del libro (p\'ags. 112-119). \\
    \large{\sc Secuencias:} \\
    Profesor:    & Explicaci\'on de la importancia de las medidas de tendencia
    central: media, moda y mediana. Explicaci\'on del procedimiento para 
    calcular las medidas de tendencia central. \\
    Alumnos:     & C\'alculo de las medidas de tendencia central de un grupo de 
    n\'umeros. Resoluci\'on de problemas que implican el uso de medidas de
    tendencia central. \\
    Profesor:   & Resoluci\'on de dudas.
\\ \hline \\
    \textsc{Tiempo:}           & 6 sesiones: 8--12 y 15 de febrero. \\
    \textsc{Temas y subtemas:} & \'Algebra: ecuaciones lineales con una
    inc\'ognita. \\
    \sc{Evidencias a evaluar:} & 5 trabajos en libreta, 3 tareas y lecci\'on 19 
    del libro (p\'ags. 96-99). \\
    \large{\sc Secuencias:} \\
    Profesor:    & Explicaci\'on del concepto de ecuaci\'on. Explicaci\'on de 
    situaciones cotidianas en las que se requiere la resoluci\'on de ecuaciones.
    Recapitulaci\'on del proceso de supresi\'on de t\'erminos semejantes. \\
    Alumnos:    & Reducci\'on de t\'erminos semejantes de expresiones 
    algebraicas simples.\\
    Profesor:   & Explicaci\'on de los proceso de resoluci\'on de ecuaciones 
    mediante tanteo y despeje. \\
    Alumnos:    & Resoluci\'on de ecuaciones lineales mediante el m\'etodo de
    tanteo. Resoluci\'on de ecuaciones lineales mediante el m\'etodo de 
    despeje. Resolici\'on de problemas que implican el uso de ecuaciones 
    lineales.
\\ \hline \\
    \textsc{Tiempo:}           & 4 sesiones: 16--19  de febrero. \\
    \textsc{Temas y subtemas:} & M\'inimo com\'un m\'ultiplo (MCM) y m\'aximo com\'un 
    divisor (MCD): problemas. \\
    \sc{Evidencias a evaluar:} & 2 trabajos en libreta y 2 tareas. \\
    \large{\sc Secuencias:} \\
    Profesor:    & Recordatorio del procedimiento para obtener el MCM y MCD de 
    un grupo de n\'umeros. \\
    Alumnos:     & C\'alculo del MCM y MCD de distintos grupos de n\'umeros 
    dados. \\
    Profesor:   & Planteamiento de problemas en los que se requiere el c\'alculo 
    del MCM. \\
    Alumnos:    & Resoluci\'on de problemas que implican el c\'alculo del MCM. \\
    Profesor:   & Planteamiento de problemas en los que se requiere el c\'alculo 
    del MCD. \\
    Alumnos:    & Resoluci\'on de problemas que implican el c\'alculo del MCD. 
\\ \hline \\
\end{tabular}

\newpage

\section{Temas  y secuencias did\'acticas (continuaci\'on)}

\begin{tabular}[t]{R{0.25\textwidth}p{0.71\textwidth}}

    \textsc{Tiempo:}           & 4 sesiones: 22--25  de febrero. \\
    \textsc{Temas y subtemas:} & Repaso para examen bimestral. \\
    \sc{Evidencias a evaluar:} & 1 trabajo: repaso para examen bimestral. \\
    \large{\sc Secuencias:} \\
    Profesor:   & Entrega del cuadernillo de repaso para examen bimestral desde
    el viernes 19 de febrero. \\
    Alumnos:    & Durante el fin de semana (19-21 de febrero) los alumnos 
    deber\'an aventajar la resoluci\'on del cuadernillo de trabajo a fin de 
    presentar sus primeras dudas el lunes 22. \\ & Resoluci\'on de los 
    ejercicios correspondientes del repaso para examen. \\
    Profesor:   & Revisi\'on constante del avance de cada alumno. Resoluci\'on 
    de dudas de los alumnos y ayuda grupal en temas que lo requieran. 
\\ \hline \\

    \textsc{Tiempo:}& 2 sesiones: viernes 26 y lunes 29  de febrero. \\
            & {\Large \sc Aplicaci\'on de examen bimestral} (26 de febrero). \\           
            & Revisi\'on grupal del los resultados del examen bimestral y
            correcci\'on de examen (29 de febrero).\\
\\ \hline


\end{tabular}




\end{document}


\section{Informaci\'on  y secuencias did\'acticas}
\begin{tabular}[t]{R{0.25\textwidth}p{0.71\textwidth}}
    \textsc{Tiempo:}                    &1 sesi\'on: 4 de agosto. \\
    \textsc{Temas y subtemas:}          &Presentaci\'on del profesor y de las reglas generales en el salón de clase. \\
    \textsc{Aprendizaje esperado: }     &El alumno retoma y reafirma conocimentos para realizar operaciones basicas con decimales. \\
    \textsc{Evidencias a evaluar:}      &4 tareas y 3 ex\'amenes rapidos. \\\\
    \textsc{\large Secuencias:} \\
    Alumno      &Resoluci\'on de problemas complementarios referentes a operaciones con decimales. \\
    Profesor    &Aplicaci\'on de ex\'amenes r\'apidos para evaluar el el aprendizaje de operaciones con n\'umeros decimales. \\
                &Retroalimentaci\'on personal acerca de las fallas en dichos temas. \\
    Alumno      &Correcci\'on de errores. \\
    Profesor    &Revisi\'on del proceso de aprendizaje. \\ 
\\ \hline \\
\end{tabular}
